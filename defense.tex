\documentclass[12pt]{beamer}

\usetheme[progressbar=frametitle]{metropolis}
\usepackage{appendixnumberbeamer}
\usepackage{csquotes}
\usepackage{booktabs}
\usepackage[scale=2]{ccicons}
\usepackage{pgfplots}
\usepgfplotslibrary{dateplot}
\usepackage{xspace}
\newcommand{\themename}{\textbf{\textsc{metropolis}}\xspace}

\title{Significant News Detection}
\subtitle{Master's Thesis}
\date{\today}
\author{Diwas Sharma}
\institute{The University of Alabama in Huntsville}
% \titlegraphic{\hfill\includegraphics[height=1.5cm]{logo.pdf}}

\begin{document}

\maketitle

\begin{frame}{Table of contents}
  \setbeamertemplate{section in toc}[sections numbered]
  \tableofcontents[hideallsubsections]
\end{frame}

\section{Introduction}

\begin{frame}[fragile]{Motivation}
    Based on a recent study, 68 percent of US adults said they at least occasionally get news
    on social media; however, 57 percent of those people expect the news to be largely inaccurate \cite{matsa2018news}.
\end{frame}

\begin{frame}[fragile]{Definition}
    \enquote{Fake news is the deliberate presentation of (typically) false or misleading claims as news, where the claims are misleading by design.} \cite{gelfert2018fake}
\end{frame}

\begin{frame}[fragile]{Background}
\end{frame}

\begin{frame}[fragile]{Problem}
\end{frame}

\section{Background}

\begin{frame}{Metropolis titleformats}
\end{frame}

\section{Method}
\begin{frame}{Metropolis titleformats}
\end{frame}

\section{Conclusion}

\begin{frame}{Summary}

\end{frame}

{\setbeamercolor{palette primary}{fg=black, bg=yellow}
\begin{frame}[standout]
  Questions?
\end{frame}
}

\appendix

\begin{frame}[allowframebreaks]{References}
    \bibliography{references}
    \bibliographystyle{unsrt}

\end{frame}

\end{document}
